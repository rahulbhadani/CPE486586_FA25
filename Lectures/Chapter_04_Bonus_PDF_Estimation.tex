\documentclass[12pt, xcolor=dvipsnames,svgnames,x11names]{article}

\usepackage{tikz}
\usepackage{pgfplots}
\usepackage{minted}





\def\m@th{\normalcolor\mathsurround\z@}

\setlength{\parskip}{\baselineskip}%
\setlength{\parindent}{0pt}%

\def\e{{\textrm{e}}}
\def\d{\textrm{d}}
\def\T{{\textsf{T}}}
\def\sech{{\textrm{sech}}}
\def\x{{\mathbf x}}

\def\++{{$^{++}$}}

\def\Abb{{\mathbb A}}
\def\Bbb{{\mathbb B}}
\def\Cbb{{\mathbb C}}
\def\Dbb{{\mathbb D}}
\def\Ebb{{\mathbb E}}
\def\Fbb{{\mathbb F}}
\def\Gbb{{\mathbb G}}
\def\Hbb{{\mathbb H}}
\def\Ibb{{\mathbb I}}
\def\Jbb{{\mathbb J}}
\def\Kbb{{\mathbb K}}
\def\Lbb{{\mathbb L}}
\def\Mbb{{\mathbb M}}
\def\Nbb{{\mathbb N}}
\def\Obb{{\mathbb O}}
\def\Pbb{{\mathbb P}}
\def\Qbb{{\mathbb Q}}
\def\Rbb{{\mathbb R}}
\def\Sbb{{\mathbb S}}
\def\Tbb{{\mathbb T}}
\def\Ubb{{\mathbb U}}
\def\Vbb{{\mathbb V}}
\def\Wbb{{\mathbb W}}
\def\Xbb{{\mathbb X}}
\def\Ybb{{\mathbb Y}}
\def\Zbb{{\mathbb Z}}



\def\Acal{{\mathcal A}}
\def\Bcal{{\mathcal B}}
\def\Ccal{{\mathcal C}}
\def\Dcal{{\mathcal D}}
\def\Ecal{{\mathcal E}}
\def\Fcal{{\mathcal F}}
\def\Gcal{{\mathcal G}}
\def\Hcal{{\mathcal H}}
\def\Ical{{\mathcal I}}
\def\Jcal{{\mathcal J}}
\def\Kcal{{\mathcal K}}
\def\Lcal{{\mathcal L}}
\def\Mcal{{\mathcal M}}
\def\Ncal{{\mathcal N}}
\def\Ocal{{\mathcal O}}
\def\Pcal{{\mathcal P}}
\def\Qcal{{\mathcal Q}}
\def\Rcal{{\mathcal R}}
\def\Scal{{\mathcal S}}
\def\Tcal{{\mathcal T}}
\def\Ucal{{\mathcal U}}
\def\Vcal{{\mathcal V}}
\def\Wcal{{\mathcal W}}
\def\Xcal{{\mathcal X}}
\def\Ycal{{\mathcal Y}}
\def\Zcal{{\mathcal Z}}

\def\abf{{\mathbf a}}
\def\bbf{{\mathbf b}}
\def\cbf{{\mathbf c}}
\def\dbf{{\mathbf d}}
\def\ebf{{\mathbf e}}
\def\fbf{{\mathbf f}}
\def\gbf{{\mathbf g}}
\def\hbf{{\mathbf h}}
\def\ibf{{\mathbf i}}
\def\jbf{{\mathbf j}}
\def\kbf{{\mathbf k}}
\def\lbf{{\mathbf l}}
\def\mbf{{\mathbf m}}
\def\nbf{{\mathbf n}}
\def\obf{{\mathbf o}}
\def\pbf{{\mathbf p}}
\def\qbf{{\mathbf q}}
\def\rbf{{\mathbf r}}
\def\sbf{{\mathbf s}}
\def\tbf{{\mathbf t}}
\def\ubf{{\mathbf u}}
\def\vbf{{\mathbf v}}
\def\wbf{{\mathbf w}}
\def\xbf{{\mathbf x}}
\def\ybf{{\mathbf y}}
\def\zbf{{\mathbf z}}

\def\Abf{{\mathbf A}}
\def\Bbf{{\mathbf B}}
\def\Cbf{{\mathbf C}}
\def\Dbf{{\mathbf D}}
\def\Ebf{{\mathbf E}}
\def\Fbf{{\mathbf F}}
\def\Gbf{{\mathbf G}}
\def\Hbf{{\mathbf H}}
\def\Ibf{{\mathbf I}}
\def\Jbf{{\mathbf J}}
\def\Kbf{{\mathbf K}}
\def\Lbf{{\mathbf L}}
\def\Mbf{{\mathbf M}}
\def\Nbf{{\mathbf N}}
\def\Obf{{\mathbf O}}
\def\Pbf{{\mathbf P}}
\def\Qbf{{\mathbf Q}}
\def\Rbf{{\mathbf R}}
\def\Sbf{{\mathbf S}}
\def\Tbf{{\mathbf T}}
\def\Ubf{{\mathbf U}}
\def\Vbf{{\mathbf V}}
\def\Wbf{{\mathbf W}}
\def\Xbf{{\mathbf X}}
\def\Ybf{{\mathbf Y}}
\def\Zbf{{\mathbf Z}}

\def\phat{{\hat p}}

\usepackage{amsmath}
\DeclareMathOperator*{\argmax}{arg\,max}
\DeclareMathOperator*{\argmin}{arg\,min}

\def\xbar{{\overline x}}
\def\ybar{{\overline y}}
\def\e{{\textrm{e}}}
\def\d{\textrm{d}}
\def\T{{\textsf{T}}}
\def\sech{{\textrm{sech}}}
\def\x{{\mathbf x}}

\def\++{{$^{++}$}}

\def\Abb{{\mathbb A}}
\def\Bbb{{\mathbb B}}
\def\Cbb{{\mathbb C}}
\def\Dbb{{\mathbb D}}
\def\Ebb{{\mathbb E}}
\def\Fbb{{\mathbb F}}
\def\Gbb{{\mathbb G}}
\def\Hbb{{\mathbb H}}
\def\Ibb{{\mathbb I}}
\def\Jbb{{\mathbb J}}
\def\Kbb{{\mathbb K}}
\def\Lbb{{\mathbb L}}
\def\Mbb{{\mathbb M}}
\def\Nbb{{\mathbb N}}
\def\Obb{{\mathbb O}}
\def\Pbb{{\mathbb P}}
\def\Qbb{{\mathbb Q}}
\def\Rbb{{\mathbb R}}
\def\Sbb{{\mathbb S}}
\def\Tbb{{\mathbb T}}
\def\Ubb{{\mathbb U}}
\def\Vbb{{\mathbb V}}
\def\Wbb{{\mathbb W}}
\def\Xbb{{\mathbb X}}
\def\Ybb{{\mathbb Y}}
\def\Zbb{{\mathbb Z}}



\def\Acal{{\mathcal A}}
\def\Bcal{{\mathcal B}}
\def\Ccal{{\mathcal C}}
\def\Dcal{{\mathcal D}}
\def\Ecal{{\mathcal E}}
\def\Fcal{{\mathcal F}}
\def\Gcal{{\mathcal G}}
\def\Hcal{{\mathcal H}}
\def\Ical{{\mathcal I}}
\def\Jcal{{\mathcal J}}
\def\Kcal{{\mathcal K}}
\def\Lcal{{\mathcal L}}
\def\Mcal{{\mathcal M}}
\def\Ncal{{\mathcal N}}
\def\Ocal{{\mathcal O}}
\def\Pcal{{\mathcal P}}
\def\Qcal{{\mathcal Q}}
\def\Rcal{{\mathcal R}}
\def\Scal{{\mathcal S}}
\def\Tcal{{\mathcal T}}
\def\Ucal{{\mathcal U}}
\def\Vcal{{\mathcal V}}
\def\Wcal{{\mathcal W}}
\def\Xcal{{\mathcal X}}
\def\Ycal{{\mathcal Y}}
\def\Zcal{{\mathcal Z}}

\def\abf{{\mathbf a}}
\def\bbf{{\mathbf b}}
\def\cbf{{\mathbf c}}
\def\dbf{{\mathbf d}}
\def\ebf{{\mathbf e}}
\def\fbf{{\mathbf f}}
\def\gbf{{\mathbf g}}
\def\hbf{{\mathbf h}}
\def\ibf{{\mathbf i}}
\def\jbf{{\mathbf j}}
\def\kbf{{\mathbf k}}
\def\lbf{{\mathbf l}}
\def\mbf{{\mathbf m}}
\def\nbf{{\mathbf n}}
\def\obf{{\mathbf o}}
\def\pbf{{\mathbf p}}
\def\qbf{{\mathbf q}}
\def\rbf{{\mathbf r}}
\def\sbf{{\mathbf s}}
\def\tbf{{\mathbf t}}
\def\ubf{{\mathbf u}}
\def\vbf{{\mathbf v}}
\def\wbf{{\mathbf w}}
\def\xbf{{\mathbf x}}
\def\ybf{{\mathbf y}}
\def\zbf{{\mathbf z}}

\def\Abf{{\mathbf A}}
\def\Bbf{{\mathbf B}}
\def\Cbf{{\mathbf C}}
\def\Dbf{{\mathbf D}}
\def\Ebf{{\mathbf E}}
\def\Fbf{{\mathbf F}}
\def\Gbf{{\mathbf G}}
\def\Hbf{{\mathbf H}}
\def\Ibf{{\mathbf I}}
\def\Jbf{{\mathbf J}}
\def\Kbf{{\mathbf K}}
\def\Lbf{{\mathbf L}}
\def\Mbf{{\mathbf M}}
\def\Nbf{{\mathbf N}}
\def\Obf{{\mathbf O}}
\def\Pbf{{\mathbf P}}
\def\Qbf{{\mathbf Q}}
\def\Rbf{{\mathbf R}}
\def\Sbf{{\mathbf S}}
\def\Tbf{{\mathbf T}}
\def\Ubf{{\mathbf U}}
\def\Vbf{{\mathbf V}}
\def\Wbf{{\mathbf W}}
\def\Xbf{{\mathbf X}}
\def\Ybf{{\mathbf Y}}
\def\Zbf{{\mathbf Z}}

\def\phat{{\hat p}}
\def\what{{\hat w}}
\def\yhat{{\hat y}}


\newcommand{\answer}{
    \fcolorbox{orange}{PapayaWhip}{
\begin{minipage}{\textwidth}
        \textbf{Answer}
    \end{minipage}
}
}

\newcommand{\codeoutput}[1]{
Output:\\\vspace{5px}
    \fcolorbox{orange}{WhiteSmoke}{
\begin{minipage}{\textwidth}
        { \color{DodgerBlue} 
        
        
       \texttt{#1}
       }
    \end{minipage}
}
}


\newcommand{\orangebox}[1]{
    \fcolorbox{orange}{white}{
\begin{minipage}{\textwidth}
       {\color{MidnightBlue} 
       #1
       }
    \end{minipage}
}
}


\definecolor{lightapricot}{rgb}{0.99, 0.84, 0.69}
\definecolor{lightblue}{rgb}{0.68, 0.85, 0.90}
\definecolor{blanchedalmond}{rgb}{1.0, 0.92, 0.8}
\definecolor{blizzardblue}{rgb}{0.67, 0.9, 0.93}
\definecolor{blond}{rgb}{0.98, 0.94, 0.75}
\definecolor{babyblueeyes}{rgb}{0.63, 0.79, 0.95}
\definecolor{babypink}{rgb}{0.96, 0.76, 0.76}
\definecolor{bananamania}{rgb}{0.98, 0.91, 0.71}
\definecolor{beaublue}{rgb}{0.74, 0.83, 0.9}
\definecolor{antiquewhite}{rgb}{0.98, 0.92, 0.84}
\definecolor{anti-flashwhite}{rgb}{0.95, 0.95, 0.96}
\definecolor{brightlavender}{rgb}{0.75, 0.58, 0.89}
\definecolor{brightube}{rgb}{0.82, 0.62, 0.91}
\definecolor{brilliantlavender}{rgb}{0.96, 0.73, 1.0}
\definecolor{lightcerulean}{rgb}{0.11, 0.67, 0.84}
\definecolor{cpeyellow}{HTML}{F5F5DC}                 % Beige (much softer)
\definecolor{powderGreen}{HTML}{F0F8E8}               % Very pale mint


% Darker versions of the original colors
\definecolor{deepapricot}{rgb}{0.85, 0.65, 0.45}        % darker lightapricot
\definecolor{deepblue}{rgb}{0.45, 0.65, 0.75}           % darker lightblue
\definecolor{toastedalmond}{rgb}{0.85, 0.75, 0.60}      % darker blanchedalmond
\definecolor{stormblue}{rgb}{0.45, 0.70, 0.75}          % darker blizzardblue
\definecolor{goldenrod}{rgb}{0.80, 0.75, 0.50}          % darker blond
\definecolor{royalblue}{rgb}{0.40, 0.60, 0.80}          % darker babyblueeyes
\definecolor{rosewood}{rgb}{0.75, 0.45, 0.45}           % darker babypink
\definecolor{mustardyellow}{rgb}{0.80, 0.70, 0.45}      % darker bananamania
\definecolor{navyblue}{rgb}{0.50, 0.60, 0.70}           % darker beaublue
\definecolor{antiquetan}{rgb}{0.80, 0.70, 0.60}         % darker antiquewhite
\definecolor{silvergray}{rgb}{0.75, 0.75, 0.80}         % darker anti-flashwhite
\definecolor{darklavender}{rgb}{0.55, 0.35, 0.70}       % darker brightlavender
\definecolor{deepube}{rgb}{0.65, 0.40, 0.75}            % darker brightube
\definecolor{plumviolet}{rgb}{0.75, 0.50, 0.85}         % darker brilliantlavender
\definecolor{deepcerulean}{rgb}{0.08, 0.45, 0.65}       % darker lightcerulean

\definecolor{roseRed}{HTML}{e20047}
\definecolor{coolGray}{HTML}{474747} % Neutral gray to contrast roseRed
\definecolor{mocha}{HTML}{a47864}
\definecolor{sage}{HTML}{8a9a5b}
\definecolor{dustyblue}{HTML}{6e8ca0}
\definecolor{terracotta}{HTML}{c87f5b}
\definecolor{lavender}{HTML}{9d8bb0}
\definecolor{darkmocha}{HTML}{7a5a4b}
\definecolor{darksage}{HTML}{667244}
\definecolor{darkdustyblue}{HTML}{526878}
\definecolor{darkterracotta}{HTML}{9e6347}
\definecolor{darklavender}{HTML}{766688}
\definecolor{winery}{HTML}{7e212a}

\usepackage{sectsty}
\sectionfont{\color{darklavender}}  % sets colour of sections
\subsectionfont{\color{plumviolet}}  % sets colour of sections
\subsubsectionfont{\color{deepube}}  % sets colour of sections

\usepackage{xcolor}
%\usepackage{universityos}
\usepackage{xspace}
\usepackage{url}
\usepackage[colorlinks=true,bookmarks=false,linkcolor=black,urlcolor=blue,citecolor=black]{hyperref}
\usepackage{fancyhdr}
\usepackage[top=1in,bottom=1in,left=1in,right=1in]{geometry}
\usepackage{multicol}
\usepackage{pgfplots}
\usetikzlibrary{circuits.logic.US}
% \usepackage{tgschola}
\let\temp\rmdefault
\usepackage{mathpazo}
\let\rmdefault\temp

\usepackage{eso-pic}



\usetikzlibrary{calc}
\usepackage{circuitikz}



\usepackage[many]{tcolorbox}
\setlength{\parindent}{0pt}
\setlength{\parskip}{4pt}%

\newcommand{\coursenumber}{CPE 486/586}
\newcommand{\courseyear}{2025\xspace}
\newcommand{\coursedatetime}{Tue/Thur 11:20 AM -- 12:40 PM\xspace}
\newcommand{\coursetitle}{Machine Learning for Engineering Applications\xspace}
\newcommand{\courseterm}{Fall, \courseyear}
\newcommand{\instructorname}{Rahul Bhadani\xspace}
\newcommand{\instructoremail}{\href{mailto:rahul.bhadani@uah.edu}{rahul.bhadani@uah.edu}}
\newcommand{\instructoroffice}{ENG 217-H\xspace}


\renewcommand{\familydefault}{\sfdefault}


% Redefine \maketitle to include tcolorbox
\makeatletter
\renewcommand{\maketitle}{%
      \begin{center}
         \begin{tcolorbox}[
               enhanced,
               colback=white,
               colframe=stormblue,
               boxrule=1pt,
               arc=3pt,
               left=0pt,
               right=0pt,
               top=0pt,
               bottom=0pt,
               drop shadow={goldenrod!25!white},
         ]
               % Header section
               \begin{tcolorbox}[
                  enhanced,
                  colback=goldenrod,
                  colframe=goldenrod,
                  boxrule=0pt,
                  arc=3pt,
                  sharp corners=south,
                  left=2em,
                  right=2em,
                  top=1em,
                  bottom=1em
               ]
                  \centering
                  {\LARGE\bfseries\textcolor{white}{\@title}}
               \end{tcolorbox}
               
               % Content section
               \vspace{1em}
               \centering
               {\large\textcolor{darkgray}{\@author}} \\[0.8em]
               {\normalsize\textcolor{darkgray}{\@date}}
               \vspace{1em}
         \end{tcolorbox}
      \end{center}
      \vspace{2em}
}
\makeatother



\newcommand{\hw}{Probability Density Function Estimation: Theory and Mathematics}
\newcommand{\duedate}{Sept 14, 2025}
\newcommand{\hwturnin}{hw01}

\newenvironment{multiequation}[1][]{%
\begin{equation}%
   \begin{aligned}%
   \ifx#1\@empty\else\label{#1}\fi%
}{%
   \end{aligned}%
\end{equation}%
}
\pagestyle{fancy}

\lhead{\footnotesize{\coursenumber, \courseterm}, \footnotesize{UAH}}
\chead{}
\rhead{\footnotesize{\emph{\hw}}}
\lfoot{\footnotesize{\instructorname}}
\cfoot{\footnotesize{\thepage}}
\rfoot{\footnotesize{\textit{Last Revised:~\today}}}
\renewcommand{\headrulewidth}{0.1pt}
\renewcommand{\footrulewidth}{0.1pt}
\newcommand{\minipagewidth}{0.8\columnwidth}

\renewcommand{\thefootnote}{\fnsymbol{footnote}}


\renewcommand{\headrulewidth}{0.1pt}
\renewcommand{\footrulewidth}{0.1pt}

\renewcommand{\thefootnote}{\fnsymbol{footnote}}


\usepackage[utf8]{inputenc}
\usepackage{amsmath,amsfonts,amssymb,amsthm}
\usepackage{geometry}
\usepackage{graphicx}
\usepackage{hyperref}
\usepackage{natbib}
\usepackage{setspace}
\usepackage{fancyhdr}
\usepackage{titlesec}
\usepackage{enumitem}
\usepackage{listings}
\usepackage{xcolor}

\geometry{margin=1in}
\onehalfspacing
\pagestyle{fancy}
\fancyhf{}
\rhead{\thepage}
\lhead{Probability Density Function Estimation}

\titleformat{\section}{\large\bfseries}{\thesection}{1em}{}
\titleformat{\subsection}{\normalsize\bfseries}{\thesubsection}{1em}{}
\titleformat{\subsubsection}{\normalsize\itshape}{\thesubsubsection}{1em}{}

\newtheorem{theorem}{Theorem}
\newtheorem{definition}{Definition}
\newtheorem{lemma}{Lemma}
\newtheorem{proposition}{Proposition}

% Code listing style
\lstset{
    basicstyle=\ttfamily\footnotesize,
    breaklines=true,
    frame=single,
    numbers=left,
    numberstyle=\tiny,
    backgroundcolor=\color{gray!10}
}

\begin{document}


\title{\textsc{\hw\\
\coursenumber}}
\author{Instructor: \instructorname}
\date{\duedate}

\maketitle

\tableofcontents

\section{Introduction and Problem Formulation}

\subsection{Problem Statement}

Given a time series of room temperature measurements $X_1, X_2, \ldots, X_n$ collected every 5 minutes from 9 AM to 9 PM, we want to estimate the probability density function $f(x)$ that describes the distribution of temperature values.

\subsection{Mathematical Framework}

Let $X$ be a continuous random variable representing room temperature. The probability density function $f(x)$ satisfies:
\begin{align}
f(x) &\geq 0 \text{ for all } x \\
\int_{-\infty}^{\infty} f(x)dx &= 1 \\
P(a \leq X \leq b) &= \int_a^b f(x)dx
\end{align}

Our goal is to estimate $f(x)$ from the observed sample $\{x_1, x_2, \ldots, x_n\}$.

\section{Data Structure and Sampling Considerations}

\subsection{Temporal Structure}

With measurements every 5 minutes from 9 AM to 9 PM:
\begin{itemize}
\item Time points: $t_i = 9:00 + 5i$ minutes, $i = 0, 1, \ldots, 143$
\item Daily sample size: $n = 144$ observations
\item For $k$ days of data: $N = 144k$ total observations
\end{itemize}

\subsection{Stationarity Assumptions}

The basic PDF estimation assumes that observations are independent and identically distributed (i.i.d.). However, room temperature data may exhibit:
\begin{itemize}
\item \textbf{Temporal dependence}: $X(t)$ may be correlated with $X(t-1)$
\item \textbf{Diurnal patterns}: $f(x|t)$ may vary with time of day
\item \textbf{Non-stationarity}: Distribution parameters may change over time
\end{itemize}

\textbf{However, in the most simplest case, we will assume i.i.d.}
\section{Parametric PDF Estimation}

\subsection{Normal Distribution Model}

If we assume $X \sim N(\mu, \sigma^2)$, the PDF is:
\begin{equation}
f(x; \mu, \sigma^2) = \frac{1}{\sigma\sqrt{2\pi}} \exp\left(-\frac{(x-\mu)^2}{2\sigma^2}\right)
\end{equation}

\subsubsection{Parameter Estimation using Maximum Likelihood}

The likelihood function for $n$ observations is:
\begin{equation}
L(\mu, \sigma^2) = \prod_{i=1}^n f(x_i; \mu, \sigma^2)
\end{equation}

The log-likelihood is:
\begin{equation}
\ell(\mu, \sigma^2) = -\frac{n}{2} \ln(2\pi) - \frac{n}{2} \ln(\sigma^2) - \frac{1}{2\sigma^2} \sum_{i=1}^n (x_i - \mu)^2
\end{equation}

\textbf{Maximum Likelihood Estimators:}
\begin{align}
\hat{\mu} &= \bar{x} = \frac{1}{n} \sum_{i=1}^n x_i \\
\hat{\sigma}^2 &= \frac{1}{n} \sum_{i=1}^n (x_i - \bar{x})^2
\end{align}

\subsection{Other Parametric Distributions}

\subsubsection{Beta Distribution (for bounded temperature ranges)}
\begin{equation}
f(x; \alpha, \beta) = \frac{x^{\alpha-1}(1-x)^{\beta-1}}{B(\alpha,\beta)} \text{ for } x \in [0,1]
\end{equation}

After rescaling temperature to $[0,1]$ using: $x' = \frac{x - x_{\min}}{x_{\max} - x_{\min}}$

\subsubsection{Gamma Distribution (for positive temperatures in Kelvin)}
\begin{equation}
f(x; \alpha, \beta) = \frac{\beta^\alpha}{\Gamma(\alpha)} x^{\alpha-1} e^{-\beta x} \text{ for } x > 0
\end{equation}

\subsection{Goodness-of-Fit Testing}

\subsubsection{Kolmogorov-Smirnov Test}
Test statistic: 
\begin{equation}
D_n = \sup_x|F_n(x) - F_0(x)|
\end{equation}

Where $F_n(x)$ is the empirical CDF and $F_0(x)$ is the theoretical CDF.

\subsubsection{Anderson-Darling Test}
\begin{equation}
A  = -n - \sum_{i=1}^n \frac{2i-1}{n}\bigg[ \ln F(x_{(i)}) + \ln \bigg( 1-  F(x_{(n+1-i)}) \bigg) \bigg]
\end{equation}

Source: \url{https://www.statsref.com/HTML/anderson-darling.html}

\section{Non-Parametric PDF Estimation}

\subsection{Kernel Density Estimation (KDE)}

The kernel density estimator is:
\begin{equation}
\hat{f}(x) = \frac{1}{nh} \sum_{i=1}^n K\left(\frac{x - x_i}{h}\right)
\end{equation}

Where:
\begin{itemize}
\item $K(u)$ is the kernel function
\item $h$ is the bandwidth parameter
\item $n$ is the sample size
\end{itemize}

\underline{Source:} Chapter 4, Harrou, Fouzi, Abdelhafid Zeroual, Mohamad Mazen Hittawe, and Ying Sun. Road traffic modeling and management: Using statistical monitoring and deep learning. Elsevier, 2021.

\subsubsection{Common Kernel Functions}

\textbf{Gaussian Kernel:}
\begin{equation}
K(u) = \frac{1}{\sqrt{2\pi}} \exp\left(-\frac{u^2}{2}\right)
\end{equation}

\textbf{Epanechnikov Kernel (optimal):}
\begin{equation}
K(u) = \frac{3}{4}(1 - u^2) \text{ for } |u| \leq 1, \text{ } 0 \text{ otherwise}
\end{equation}

\textbf{Uniform Kernel:}
\begin{equation}
K(u) = \frac{1}{2} \text{ for } |u| \leq 1, \text{ } 0 \text{ otherwise}
\end{equation}

\subsection{Bandwidth Selection}

\subsubsection{Silverman's Rule of Thumb}
\begin{equation}
h = 0.9 \times \min(\hat{\sigma}, \text{IQR}/1.34) \times n^{-1/5}
\end{equation}

Where IQR is the interquartile range.

\subsubsection{Cross-Validation Bandwidth}
Minimize the integrated squared error:
\begin{equation}
\text{ISE}(h) = \int[\hat{f}(x) - f(x)]^2 dx
\end{equation}

\textbf{Practical Cross-Validation:}
\begin{equation}
h^* = \arg\min \sum_{i=1}^n \int[\hat{f}_{-i}(x) - \delta(x - x_i)]^2 dx
\end{equation}

Where $\hat{f}_{-i}(x)$ is the KDE excluding observation $x_i$.

\subsection{Histogram-Based Estimation}

\subsubsection{Basic Histogram}
\begin{equation}
\hat{f}(x) = \frac{n_j}{n \times \Delta x} \text{ for } x \in [x_j, x_{j+1})
\end{equation}

Where $n_j$ is the count in bin $j$ and $\Delta x$ is the bin width.

\subsubsection{Optimal Bin Width (Sturges' Rule)}
Number of bins: $k = \lceil 1 + \log_2(n) \rceil$

Bin width: $\Delta x = \frac{x_{\max} - x_{\min}}{k}$

\subsubsection{Freedman-Diaconis Rule}
\begin{equation}
\Delta x = 2 \times \text{IQR} \times n^{-1/3}
\end{equation}

\section{Time-Varying PDF Models}

\subsection{Conditional PDF Estimation}

For non-stationary data, estimate $f(x|t)$:

\subsubsection{Kernel Regression Approach}
\begin{equation}
\hat{f}(x|t) = \sum_{i=1}^n w_i(t) K\left(\frac{x - x_i}{h}\right)
\end{equation}

Where $w_i(t)$ are time-dependent weights:
\begin{equation}
w_i(t) = \frac{W\left(\frac{t - t_i}{b}\right)}{\sum_{j=1}^n W\left(\frac{t - t_j}{b}\right)}
\end{equation}

\subsection{Functional Data Analysis}

Model temperature as a function $T(t)$ and estimate the distribution of functional parameters.

\subsubsection{Fourier Representation}
\begin{equation}
T(t) = \mu(t) + \sum_{k=1}^K [a_k\cos(2\pi kt/P) + b_k\sin(2\pi kt/P)] + \varepsilon(t)
\end{equation}

Where $P = 12$ hours is the period.

\section{Model Selection and Validation}

\subsection{Information Criteria}

\subsubsection{Akaike Information Criterion}
\begin{equation}
\text{AIC} = -2\ell(\hat{\theta}) + 2p
\end{equation}

\subsubsection{Bayesian Information Criterion}
\begin{equation}
\text{BIC} = -2\ell(\hat{\theta}) + p \ln(n)
\end{equation}

Where $p$ is the number of parameters.

\subsection{Cross-Validation}

\subsubsection{K-Fold Cross-Validation}
\begin{enumerate}
\item Divide data into $K$ folds
\item For each fold $k$, estimate $\hat{f}_{-k}(x)$ using remaining data
\item Evaluate log-likelihood on fold $k$: $\text{LL}_k = \sum_{i \in k} \ln(\hat{f}_{-k}(x_i))$
\item Average: $\text{CV-LL} = \frac{1}{K} \sum_{k=1}^K \text{LL}_k$
\end{enumerate}

\subsection{Bootstrap Confidence Intervals}

\subsubsection{Bootstrap Procedure}
\begin{enumerate}
\item Resample with replacement: $X^* = \{x_1^*, x_2^*, \ldots, x_n^*\}$
\item Compute $\hat{f}^*(x)$
\item Repeat $B$ times
\item Construct pointwise confidence intervals from bootstrap distribution
\end{enumerate}

\section{Practical Considerations}

\subsection{Sample Size Requirements}

For reliable PDF estimation:
\begin{itemize}
\item Parametric: $n \geq 30$ typically sufficient
\item Non-parametric KDE: $n \geq 100$ recommended
\item Complex time-varying models: $n \geq 1000$ may be needed
\end{itemize}

\subsection{Boundary Effects}

Room temperatures are bounded (e.g., 15-30°C). Use:
\begin{itemize}
\item Boundary kernels for KDE
\item Transformed variables
\item Reflection methods
\end{itemize}

\subsection{Computational Complexity}
\begin{itemize}
\item Histogram: $O(n)$
\item KDE: $O(n^2)$ for evaluation at $n$ points
\item Parametric MLE: $O(n)$ per iteration
\end{itemize}

\section{Implementation Algorithm}

\subsection{General Workflow}

\begin{lstlisting}[caption=General Workflow for PDF Estimation]
1. Data Preprocessing:
   - Remove outliers (|x - median| > 3xMAD)
   - Check for temporal patterns
   - Test for stationarity

2. Model Selection:
   - Fit parametric candidates
   - Compute non-parametric estimates
   - Use cross-validation for comparison

3. Parameter Estimation:
   - If parametric: MLE or method of moments
   - If KDE: optimize bandwidth
   - If histogram: optimize bin width

4. Validation:
   - Goodness-of-fit tests
   - Cross-validation
   - Bootstrap confidence intervals

5. Final Estimation:
   - Select best model
   - Provide uncertainty quantification
\end{lstlisting}

\subsection{Error Metrics}

\subsubsection{Mean Integrated Squared Error}
\begin{equation}
\text{MISE} = E\left[\int(\hat{f}(x) - f(x))^2 dx\right]
\end{equation}

\subsubsection{Pointwise Mean Squared Error}
\begin{equation}
\text{MSE}(x) = E[(\hat{f}(x) - f(x))^2] = \text{Bias}^2(x) + \text{Variance}(x)
\end{equation}


\end{document}